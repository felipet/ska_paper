%% Section that includes the experiments performed.

This section describes the accomplished experiments that prove how the developed system fulfills SKA's requirements for the PPS distribution system. The first experiment is a general characterization of the developed platform. The second one, tests the equipment in a realistic network topology to evaluate the performance in a possible deployment. The last experiments evaluates the influence of some typical events, such as traffic, temperature fluctuation or \textcolor{blue}{something else} in a network deployment using the WR-ZEN.

Performance of the new WR platform is measured using multiple equipment, and obtained results have been compared to WRS's performance because of it's electronic design is considered as \ftglnote{en serio?}reference for WR technology. The list of the equipment and materials used for the experiments is the following:

\begin{itemize}
    \item Three White Rabbit Switches to simulate a typical WR network. All of them have hardware version 3.4 and 5.0 of firmware.
    \item A White Rabbit ZEN Time Provider (WR-ZEN TP) to test the performance of the new developed platform as node of a WR network. The firmware version is 1.2.
    \item Phase noise plots, ADEV, TDEV and TIE measures have been taken using a Symmetricom 3120A.
    \item The 3120A needs an external stable clock reference. That reference is also needed as input for the WR equipment when Grand Master mode is set. \ftglnote{Creo que sería necesario comentar aquí que hemos diseñado una placa para poder sacar la referencia de 10Mhz doble y un PPS, pero como lo ha hecho 7s no se como ponerlo...}A Morion MV89 Oven Controlled Crystal Oscillator (OCXO) has been used as clock reference for the experiments.
    \item Multiple components for the setup of the equipment:
    \begin{itemize}
        \item Small form-factor pluggable transceptors (SFPs) to stablish the link between WR devices.  \textcolor{blue}{\textit{TODO:} Completar cuando se hayan usado}
        \item Optical fiber links. \textcolor{blue}{\textit{TODO:} Completar con las fibras que se hayan usado}
        \item \textcolor{blue}{\textit{TODO:} Something else?}
    \end{itemize}
    \item \textcolor{blue}{\textit{TODO:} Completar conforme se hagan medidas}
\end{itemize}

\subsection{Characterization of the WR-ZEN platform}
\label{subsec: charact_zen}

% Aquí las pruebas típicas de caracterización

\subsection{Network experiment} %% Buscar un nombre mejor
\label{subsec: net_exp}

% Un experimento con una cadena de WRS y al final una ZEN

\subsection{Traffic influence on synchronization accuracy}

% Posible experimento metiendo tráfico a la cadena anterior

\subsection{Redundancy}

% Este no me queda muy claro, cuando toque se verá

\subsection{Thermal characterization...}

% Aquí irían el de la cámara térmica, mover las fibras para simular viento y
% chorradas por el estilo

 
