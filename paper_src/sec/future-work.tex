The developed PPS distribution system is working at the moment in the SKA 
telescope, but it is still necessary testing it in depth with real conditions. 
Most important issues needing to be evaluated as part of 
the future work are: (i) The problem with the variable wavelength for the 
current chosen SPFs and their calibration. We experimented $alpha$ deviations 
during the calibration procedure when using different SFPs pairs of the same 
manufacturer. This is an important issue to solve because it is not feasible to 
calibrate all the SFPs pairs that will be in use. Changing manufacturer could 
be a possible solution, but the cost increment may not be assumable.  (ii) The 
evaluation of the conditions where the WR equipment will be placed. It is 
important to mention some publications about that, for example \cite{Li2015a}. 
Results included in this paper reflect the importance to correct the effects of 
a variable external temperature in order to maintain the high level of accuracy 
expected. If the final location of the equipment is not going to be fully 
controlled, development of a model which helps correcting thermal drift would 
be necessary.