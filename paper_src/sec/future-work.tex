In spite of the correct functioning of the developed PPS distribution system, 
two aspects must be considered to enhance and improve the functioning of the approach presented in this paper. 
On one hand, the utilisation of different SFPs that present variable wavelengths causing calibration problems.
On the other hand, the evaluation of the timing system in respect to diverse atmospheric conditions where the WR equipment will be finally placed in the SKA infrastructure in order to detect whether temperature changes and wind variations are able or not to degrade significantly the PPS accuracy. 

Another challenging future research topic is the improvement of the jitter of the frequency dissemination by means of WR technology. Currently, it is not possible to achieve the 1 ps jitter specification but some promising results make allow using WR for this purpose.

Moreover, the solution is flexible enough to be used in another scientific projects such as Cherenkov Telescope Array \cite{cta:website} (CTA), IFMIF-DONES \cite{ifmif:website} \cite{ifmif:ieee-paper1} or KM3NET \cite{km3net:website} \cite{km3net:tipp14}. 
