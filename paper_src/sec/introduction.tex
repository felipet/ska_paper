The DAQ \cite{daq:book1} systems are responsible for converting the analog environment signals into digitial domain and send this data to a computer. Then, it can be processed to perform several tasks such as analysis, monitoring and control. Nowadays, there are many industrial \cite{daq:res} and scientific applications \cite{daq:sensor-networks} that require a DAQ system. However, these applications are normally composed of several nodes that are located at different places. The distributed DAQ systems must be implemented in these cases. These kind of systems need a computer network to allow data sharing between different nodes. The distributed DAQ systems also include a specific node that is responsible for receiving all data, process it and extract useful information about a certain event. It is important to note it is necessary to define some mechanisms in order to match data related with the same event coming from different nodes. One possible solution is to implement a synchronization mechanism in the distributed DAQ network sharing the time information between different nodes. Thanks to the synchronization, a event can be identified by the time it occurred.

%The DAQ systems have several sensors in different places to get environment information. A %computer network must be used in these system in order to ensure that all the data can be %shared. The main concern related to these system is the need to match information about the %same event in different nodes. 

There are many mechanisms to distribute time information and signals in distributed systems. Some of them use Global Navigation Satellite System (GNSS) receivers to get timing from Satellites (GPS, Glonass, Galileo) meanwhile others use wired protocols to provide the time reference 
through the network. The GPS approach is widely used in many systems that require accurate synchronization since various instrumentation can be easily connected to different GPS receivers. The main concern related to the GPS system is sensitive to jamming issues. However, some studies reveal that the synchronization mechanisms can help and reduce the negative effects related to the interferences \cite{NOURA2016130}. Alternatively and thanks to the packet networks popularity, the timing protocols are little by little imposing as main solution for time transfer over the network. Low performance protocols as NTP, the protocol used in Internet to synchronize the computers through the network), \cite{ntf:ntp_std} are widely used while for industrial applications the preferred protocol is the IEEE-1588v2 (known as PTPv2) \cite{ieee:ieee1588_std} \cite{itu:TG8275_1_Y_1369_1}. IEEE-1588v2 is an industrial evolution of NTP, characterized by the utilization of hardware time-stamping mechanisms that significantly improved time synchronization accuracy. Typically, NTP provides about 1 ms synchronization accuracy while PTPv2, working on very well designed time-aware networks is able to achieve about 50 ns synchronization accuracy. The IEEE-1588v2 is a candidate technology for control applications such as Smart Grid \cite{NAFI201623} where the time requirement are very strict and the synchronization is one of the key aspects to take into consideration \cite{COLAK2016396}.
In the synchronization framework, it is important to distinguish between frequency, phase and time distribution.

The first case regards to the distribution of the oscillator signal through a wire or to provide a mechanism to regenerate the clock frequency. This is usually  called syntonization. 
The phase distribution is normally implemented enconding phase information in a pulse that is transmitted every second through the wire. This mechanism is known as PPS signal and is used as a reference to identify when a new second starts.
The time distribution refers to the Coordinated Universal Time (UTC) /TAI (International Atomic Time) time information sharing between different elements in the network. It ensures that all the event are trigged at the same timing in the entire network. There are several packet-based (NTP, PTP, PTPv2) or serial protocols (NMEA) that are responsible for measuring the propagation time and exchange this information between nodes. 
A network is synchronized when takes into consideration these three elements: frequency, phase PPS) and time.

Some scientific facilities need a distributed DAQ system to perform its activity in the proper way. One example is the SKA \cite{ska:project_website} project. It is an international project to build a radio telescope tens of times more sensitive and hundreds of times faster at mapping the sky than today's best radio astronomy facilities. It will become the world's largest radio telescope. The SKA telescope is composed of several types of antennas to be sperad over long distances so a distributed DAQ system is needed for data sharing and synchronization. Once completed, it will generate data at a rate more than 10 times today’s global Internet traffic. The SKA will be used to answer fundamental questions of science and about the laws of nature and imposes a technological challenge never faced before.

In the case of SKA, different solutions for time transfer were analyzed. For instance, GPS devices provide a reference frequency (10-50 MHz), a PPS signal and a serial code to provide the time (typically based on the NMEA protocol). The standard packet-based protocols such as NTP, PTP or PTPv2 are not appropiated to fulfill the SKA's strict timing specifications (2 ns for the SKA Telescope).
In this context, out research group at the University of Granada is working as a partner in the Signal and Data Transport (SaDT) \cite{ska:sadt_website} work package. Our proposal consits on the development of a new device based on the technology called WR \cite{ohwr:wr_wiki} to disseminate a PPS signal very accurately over fibre links. WR is an evolution of IEEE-1588v2 protocol and able to distribute an absolute time reference with an accuracy of 1 ns. Equipment located at each endpoint delivers a PPS signal with its edge aligned to the start of the UTC second. Therefore, each telescope can then be phased up by observing the white light fringe on one or more bright, compact sources.

This paper is organized in several sections. The current one 
presents the introduction and the motivation. Section \ref{sec:ska} introduces the SKA Telescope project, the different network and timing requirements that evidence that industrial solutions are not suitable for SKA. Section \ref{sec:wr} focuses on the fundamentals of the WR technology. Section \ref{sec:ska-pps-system} describes the proposed end-node for SKA including the justification, architecture and software support. The synchronization performance of the proposed solution is evaluated in different scenarios with special consideration on the scalability of the WR technology for the SKA deployment and the synchronization accuracy taking into account different meteorological conditions in section \ref{sec:experiments}. Final remarks and conclusions are described in section \ref{sec:conclusion}. Last sections(\ref{sec:future-work} and \ref{sec:acknowledgments}) are described the future work and the acknowledgments.

%\textcolor{red}{La motivación es pobre. Hay que ser más preciso: qué necesita SKA, qué es lo %que no cubren otras tecnologías similares (o qué desventajas tienen) y cómo (hemos %pensado)/pensamos solucionarlo.}

%\textcolor{red}{la siguiente sección debería ser el Related Work}
