\gutinote{hacen falta referencias en la intro, no hay ninguna...}
There are many mechanisms to distribute time information and signals in distributed systems. Some of them use GNSS \gutinote{GNSS: hay que definir el acrónimo} receivers to get timing from Satellites (GPS, Glonass, 
Galileo) meanwhile others use wired protocols to provide the time reference 
through the network. In all these cases, it is important to distinguish between frequency and phase distribution.
The first case regards to the distribution of the oscillator signal through a wire or to provide a mechanism to regenerate the clock frequency. This is usually  called syntonization. On the other hand, the latter ensures that, in all the elements of a network, events are triggered exactly at the same instant of time. This process is known as synchronization. 

In phase distribution scenarios, phase information is encoded in a pulse that is transmitted every second thought the wire. This is used as a reference mechanism to identify when a new second starts. This is typically called PPS (Pulse Per Second) signal. Finally, the last problem is the provision of the time that is not only ticking at the same time, having the same reference about when to start the counting but also having the same time in all the devices. This can be distributed by propagation of the time information from a central time server and then, by measuring the propagation time of this message, annotating it in each node. A network is synchronized when takes into consideration these three elements: frequency, phase (PPS) and time. \gutinote{este párrafo hay que reescribirlo, no se entiende}

This contribution focuses on finding a solution to distribute a very accurate PPS 
signal over the Square Kilometer Array (SKA) Telescope  network
\cite{ska:project_website} and, currently, our research group is working as a partner in the Signal and Data Transport (SaDT) \cite{ska:sadt_website} work package \gutinote{me resulta rara esta parte de "nuestro grupo de investigación ha participado en SaDT...}. The SKA is an international project to build a radio 
telescope tens of times more sensitive and hundreds of times faster at mapping 
the sky than today’s best radio astronomy facilities. It will become the 
world’s largest radio telescope. But the SKA is not a single telescope, it is a 
collection of various types of antennas to be spread over long distances. Once 
completed, it will generate data at a rate more than 10 times today’s global 
Internet traffic. The SKA will be used to answer fundamental questions of 
science and about the laws of nature and imposes a technological challenge 
never faced before.

\textcolor{red}{La motivación es pobre. Hay que ser más preciso: qué necesita SKA, qué es lo que no cubren otras tecnologías similares (o qué desventajas tienen) y cómo (hemos pensado)/pensamos solucionarlo.}


This paper presents the development of a new device based on the technology called White Rabbit (WR) \cite{ohwr:wr_wiki} to disseminate a PPS signal very accurately over fiber links. This paper is organized in several sections. The current one 
presents the introduction and the motivation. Section \ref{sec:ska} introduces the SKA Telescope project, the different network and timing requirements that 
evidence that industrial solutions are not suitable for SKA. Section 
\ref{sec:wr} focuses on the fundamentals of the WR technology. Section \ref{sec:ska-pps-system} 
describes the proposed end-node for SKA including the 
justification, architecture and software support. The synchronization 
performance of the proposed solution is evaluated in different scenarios with 
special consideration on the scalability of the WR technology for the SKA 
deployment and the synchronization accuracy taking into account different 
meteorological conditions in section \ref{sec:experiments}. Final remarks and conclusions are described in 
section \ref{sec:conclusion}. Last sections(\ref{sec:future-work} and \ref{sec:acknowledgments}) are described the future work and the 
acknowledgments.

\textcolor{red}{la siguiente sección debería ser el Related Work}
