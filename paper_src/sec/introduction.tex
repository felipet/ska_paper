DAQ \cite{daq:book1} systems are responsible for converting analogue environment signals into digital values to ease the study of relevant events by obtaining data from distributed sensors. These data are sent to a computer to perform analysis, monitor and control tasks. Nowadays, there are many industrial \cite{daq:res} and scientific applications \cite{daq:sensor-networks} that require a DAQ system. However, these applications are normally composed of several nodes that are located at different places.
These scenarios impose the implementation of distributed DAQ systems to allow data sharing between different nodes. Distributed DAQ systems also include a specific node that is responsible for receiving, processing and extracting all data to extract critical information related to a certain event. Due to the distributed nature of DAQ systems, it becomes mandatory to define some mechanisms to match events data acquired in different nodes. One possible solution is to implement a synchronisation mechanism in the distributed DAQ network sharing the time information between different nodes. Time synchronisation makes possible the identification of events by the time they occurred.

There are many mechanisms to distribute time information and signals in distributed systems. Some of them use Global Navigation Satellite System (GNSS) receivers to get timing from Satellites (Global Positioning System (GPS), Global'naya Navigatsionnaya Sputnikovaya Sistema (GLONASS)  \cite{glonass:website}, Galileo \cite{gsa:galileo}) meanwhile others use wired protocols (National Marine Electronics Association (NMEA) Protocol, Inter-range instrumentation group time code (IRIG-B)) to disseminate the time reference through the network. The GPS approach is widely used in systems that require accurate synchronisation since devices can be easily connected to different GPS receivers. The main concern related to GPS systems regards to their exposure to jamming and spoofing attacks. However, some studies reveal that wire synchronisation mechanisms can help to reduce the negative effects related to the interference \cite{NOURA2016130}. Alternatively and thanks to the popularity of Ethernet packet networks, timing protocols are being imposed gradually as the main time transfer solution. Low performance protocols such as NTP \cite{ntf:ntp_std}, are widely used in standard applications while the industrial domain requires a more pricese protocol like the IEEE-1588v2 (known as PTPv2) \cite{ieee:ieee1588_std} \cite{itu:TG8275_1_Y_1369_1}. PTPv2 is an industrial evolution of NTP, characterised by the utilisation of hardware time-stamping mechanisms that significantly improves time synchronisation accuracy. Typically, NTP provides about 1 ms synchronisation accuracy while PTPv2 is able to achieve an accuracy about 50 ns. PTPv2 is a candidate technology for control applications such as Smart Grid where the time requirement are very strict and the synchronisation is one of the key aspects to take into consideration \cite{NAFI201623} \cite{COLAK2016396}.

In a synchronisation framework, it is important to distinguish between frequency, phase and time distribution.
The first case regards to the distribution of the oscillator signal through a wire or to provide a mechanism to regenerate the clock frequency. This is usually  called synchronisation. Phase distribution is normally implemented by encoding the phase information in a pulse that is transmitted every second through a wire. This mechanism is known as PPS signal and is used as a reference to identify when a new second starts.
Time distribution refers to the Coordinated Universal Time (UTC) / International Atomic (TAI) time information shared between different elements in the network. It ensures that all the events are trigged at the same time in the entire network. There are several packet-based (NTP, PTP, PTPv2) or serial protocols (NMEA, IRIG-B) that are responsible for measuring the time propagation and exchanging this information between nodes. 

Some scientific facilities need a distributed DAQ system to perform its activity in the proper way. One example is the SKA \cite{ska:project_website} project. It is an international project to build a radio telescope tens of times more sensitive and hundreds of times faster at mapping the sky than today's best radio astronomy facilities. It will become the world's largest radio telescope. The SKA Telescope is composed of several types of antennas settled over long distances forming a distributed DAQ system, hence requirering  data sharing and synchronisation networks. Once completed, it will generate data at a rate more than 10 times today's global Internet traffic. The SKA will be used to answer fundamental questions of science and about the laws of nature and imposes a technological challenge never faced before.

Different technologies and solutions were analysed for the SKA timing system. On the one hand, GPS devices provide a reference frequency (10-50 MHz), a PPS signal and a serial code to provide the time (typically based on the NMEA protocol). On the other hand, the standard packet-based protocols such as NTP, PTP or PTPv2 are not appropriated to fulfil the SKA's strict timing specifications (2 ns for the SKA Telescope).
In this context, the authors of this paper are currently working as a partner in the SKA Signal and Data Transport (SaDT) \cite{ska:sadt_website} work package. Our proposal consists in the development of a new device based on the technology called WR \cite{ohwr:wr_wiki} to disseminate a PPS signal very accurately over fibre links. WR is an evolution of the PTPv2 protocol, being able to distribute an absolute time reference with an accuracy below 1 ns. The equipment located at each endpoint delivers a PPS signal with its edge aligned to the start of the UTC second. Therefore, each telescope can then be phased up by observing the white light fringe on one or more bright, compact sources.

This paper is organised in several sections. The current one presents the introduction and the motivation. Section \ref{sec:ska} introduces the SKA Telescope project and the different network and timing requirements that evidence that current industrial solutions are not suitable for SKA. Section \ref{sec:wr} focuses on the fundamentals of the WR technology. Section \ref{sec:ska-pps-system} describes the proposed end-node for SKA including the justification, architecture and software support. Section \ref{sec:experiments} includes different scenarios to evaluate the scalability of the WR technology and the synchronisation accuracy taking into account different meteorological conditions. Final remarks and conclusions are described in section \ref{sec:conclusion}. 
