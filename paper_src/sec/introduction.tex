\klyonelnote{We have to add more information to this introduction...}

There are many mechanisms to distribute time information and signals. Some of them use GNNS receivers to get timing from the Satellites (GPS, Glonass, Galileo) meanwhile others use wired protocols to provide the time reference through the network. In all these cases is important to note that it is not the same to distribute frequency than distributing phase information. The first problem is about sending the oscillator signal through a wire or provide a mechanism to regenerate the clock frequency and usually is called \textcolor{blue}{syntonization}. On the other hand, the latter ensures that in all the elements of a network the events trigger exactly at the same instant and is known as synchronization. 

In the phase distribution scenario, the phase information is encoded on a pulse that is transmitted thought the wire periodically each second and use this as reference to know when a new second starts. This is typically called PPS (Pulse Per Second) signal. Finally, the last problem is the provision of the time that is not only ticking at the same time, having the same reference about when to start the counting but also having the same time in all the devices. This can be distributed by propagation of the time information from a central time server and then, by measuring the propagation time of this message, annotating it in each node. A network is synchronized when takes into consideration these three elements: frequency, phase (PPS) and time.

This contribution is focused on finding a solution to provide ultra-accurate PPS signal distribution for the Square Kilometer Array (SKA) Telescope \cite{ska:project_website}. It is an international project to build a radio telescope tens of times more sensitive and hundreds of times faster at mapping the sky than today’s best radio astronomy facilities. It will become the world’s largest radio telescope. But the SKA is not a single telescope, but a collection of various types of antennas to be spread over long distances. Once completed, it will generate data at a rate more than 10 times today’s global Internet traffic. The SKA will be used to answer fundamental questions of science and about the laws of nature and imposes a technological challenge never faced before. 

This article presents a new device based on the technology called White-Rabbit \cite{ohwr:wr_wiki} to be used as mechanism for ultra-accurate PPS signal distribution. Section 2 will present the SKA Telescope project, the different network and the timing requirements that explain why an industrial technology solution is not feasible for SKA. Section 3 focuses on the fundamentals of the White-Rabbit technology. Section 4 describes the proposed node as end-node for SKA Telescope including the justification, architecture and software support. The synchronization performance under different scenarios, network topologies and environmental conditions are finally presented on section 5 illustrating the goodness of the exposed solution. Final remarks are described on the section 6. The last sections are dedicated to the conclusion, future work and the acknowledgments.

\klyonelnote{Add some reference about the commercial synchronization solutions.}