This contribution presents an implementation of a high accuracy 
PPS distribution system based on the WR technology. The platform used for 
this approach is the WR-ZEN, which improves previous existing WR node designs 
due to the inclusion of the \textcolor{red}{Zynq?} FPGA-SoC platform and evolved hardware clocking 
circuitry. The former allows to run a Linux kernel adding high-level capabilities without using an external PC. The enhanced clocking circuitry can be configured using different schemas in order to look for the best synchronization quality results targeting the SKA needs. 

We have developed gateware, firmware and software aspects for the WR-ZEN to meet the SKA requirements. The gateware design includes all the logic elements for the FPGA that enables WR synchronization and Fine Delay FMC features. The firmware is responsible for controlling the gateware and WR functionalities. The software implements the high-level capabilities and incorporates a Linux kernel, custom drivers and several specific tools. 

One of the challenges of the SKA facilities is keeping the high accuracy 
synchronization through huge amount of nodes which are connected using long 
distance fibre links along dessert areas which present severe climatic 
conditions. In addition to that, SKA demands a strict timing requirements: a 
jitter in the tens of the picoseconds level and a time budget below 10 ns. 
However, not all of this budget is available for the PPS distribution system 
which needs to maintain the timing accuracy below 2 ns. The proposed developed 
system creates a ecosystem with several applications, drivers and scripts that 
eases the user interaction with the rest of the 
SKA components providing synchronization but also allowing to extend the system capabilities through additional FMC cards. \textcolor{red}{esto es bastante redundante con el anterior párrafo. Fusionarlo y ser más claro. Lo que no sea de SKA o no sea relevante a la contribución del paper, va fuera}

We have also considered the evaluation of the current PPS distribution system proposal 
in order to validate the developed solution for its inclusion in the SKA 
infrastructure. The results prove that the synchronization accuracy is bounded below 200 ps, being significally better that the 2 ns SKA requirement. In addition to a device characterization, 
the evaluation of the scalability was an important result due to the high 
number of nodes in SKA. We stated that a 2 level WR network is enough to 
synchronize all the expected nodes of the SKA system maintaining the required 
synchronization accuracy inside the SKA timing limits. Furthermore, we have 
designed several experiments simulating the SKA climatic place conditions 
looking for evaluating the thermal change influence in the propagation delay 
and its repercussion in the PPS distribution system. We observed that for a 
20ºC temperature difference the propagation delay is increased tens of 
nanoseconds (~ 96 ns) but the PPS offset keeps its accuracy with an offset that 
does not exceed the hundred of picoseconds (~ 211 ps).  \textcolor{red}{reescribir, está muy lioso todo este párrafo.}

Finally, the experimental results evidence that the proposed PPS distribution system based on WR fulfil the expected timing requireemnts for SKA.

\textcolor{red}{hay que ser mucho más directo y claro. Hemos hecho esto, esto y esto, que antes era "imposible", y ahora gracias a lo que hemos hecho es posible y además tenemos unos resultados mucho mejores que los propios requisitos de SKA. En los resultados hay que, sin llegar a parecer prepotente, dejar una clara evidencia y lo bueno que es el trabajo realizado.}

\textcolor{red}{hay una cosa generalizada dentro del paper, hay veces que da la impresión de que la ZEN es la plataforma elegida para distribuir el PPS, otras parece que no, que se va a evaluar y ya se verá. O es una cosa o es la otra, pero no de puede dejar esa ambigüedad}

\textcolor{red}{a veces parece que el paper va de SKA, y que el desarrollo ha sido un daño colateral...}


