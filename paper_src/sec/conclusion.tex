In this contribution, we have presented an implementation of a high accuracy 
PPS distribution system based on the WR technology. The chosen platform for 
this device is the WR-ZEN which improves the previous existing WR node designs 
thanks to the inclusion of the FPGA-SoC platform and evolved hardware clocking 
circuitry. Thanks to the new architecture, the develop process is easier than 
in embedded platforms because a Linux kernel is used and allows you to abstract 
specific hardware details. Moreover, the flexibility of the system has been 
improved so that many clocking schemas can be test looking for the best 
synchronization quality results targeting the SKA needs.

One of the challenges of the SKA facilities is keeping the high accuracy 
synchronization through huge amount of nodes which are connected using long 
distance fibre links along dessert areas which present severe climatic 
conditions. In addition to that, SKA demands a strict timing requirements: a 
jitter in the tens of the picoseconds level and a time budget below 10 ns. 
However, not all of this budget is available for the PPS distribution system 
which needs to maintain the timing accuracy below 2ns. The proposed developed 
system creates a ecosystem with several applications, drivers and scripts that 
eases the user interaction with the rest of the 
SKA components providing synchronization but also allowing to extend the system 
capabilities through additional FMC cards.

We have also consider to evaluate the current PPS distribution system proposal 
in order to validate the developed solution for its inclusion in the SKA 
infrastructure. The results, given by the device performance evaluation tests, 
prove that the synchronization accuracy is bounded below 200ps which is far 
away from the 2ns SKA requirements. In addition to a device characterization, 
the evaluation of the scalability was an important result due to the high 
number of nodes in SKA1. We stated that a 2 level WR network is enough to 
synchronize all the expected nodes of the SKA system maintaining the required 
synchronization accuracy inside the SKA timing limits. Furthermore, we have 
designed several experiments simulating the SKA climatic place conditions 
looking for evaluating the thermal change influence in the propagation delay 
and its repercussion in the PPS distribution system. We observed that for a 
20ºC temperature difference the propagation delay is increased tens of 
nanoseconds (~ 96 ns) but the PPS offset keeps its accuracy with an offset that 
does not exceed the hundred of picoseconds (~ 211 ps). 

Finally, we can conclude with the presented results that the proposed PPS 
distribution system based on WR fulfil the expected timing requirement for SKA 
radio-telescope infrastructure.