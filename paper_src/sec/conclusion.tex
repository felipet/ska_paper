In this contribution, we have introduced the distributed DAQ systems and its 
specific requirements. We have also reviewed the different synchronization 
mechanisms. We focus on the SKA project and we briefly describes its strict 
timing requirements. Nowadays, there is not any standard time distribution 
system that fulfils the SKA telescope needs. The WR technology is presented 
because it is based on standard protocols and its results ensure a 
sub-nanosecond accuracy. In this context, we present an implementation of a 
high accuracy PPS distribution system based on the WR technology that presents. 
This platform has been proposed to be integrated in the SKA infrastructure and 
currently, it is under a evaluation process. 

Our solution is based on the WR-ZEN board. This WR node improves the previous  
designs thanks to: (i) the inclusion of high-level software capabilities, and 
(ii) a high accuracy firmware design with an enhanced clock circuitry.
To provide the high-level capabilities, we have developed a Linux-based 
environment with some user-space tools and drivers. The firmware design 
includes all the elements to implement the WR protocol and some additional 
functionalities for FMC boards. The enhanced clocking circuitry can be 
configured using different schemas in order to look for the best 
synchronization quality results targeting the SKA needs. 

In addition to the platform development, we have also performed several 
experiments. We have compared some of the commercially available WR nodes and 
we obtained that the WR-ZEN platform offers the best results both in design 
flexibility and timing performance. Then we have proved that a two layer WR 
network does fully synchronize the entire number of nodes expected in the SKA's 
deployment. The MTIE results also evidence that a two layer WR network with the 
WR-ZEN as end node, satisfies the 2 ns time budget. Finally, we have measured 
the influence of temperature changes in the fibre link. For a 20ºC temperature 
gradient, the PPS offset experiments a variation around 2e-10 s.
All those experimental results evidence that the proposed PPS distribution 
system based on WR not fits perfectly with the SKA telescope requirements.


