\ftgnote{Qué es}The White Rabbit (WR) \ftglnote{Me parece interesante hacer referencia a que no es únicamente un protocolo. El proyecto va más allá del diseño de un protoclo extensión a PTP}\textst{protocol} \textcolor{teal}{technology is an international, collaborative project started at CERN, and then joined by other laboratories and companies. It was born as replacement technology for the timing system at CERN, but thanks to its versatility, improved performance, compared to the alternatives, and open nature of the project, it was quickly adopted by other scientific institutions. There is also a strong interest in private companies to extend WR in the \ftglnote{¿Se deberían incluir ejemplos, o quizás alguna referencia que apoye esto?} industrial world. }


\ftgnote{Qué es II: un poco más técnico}\textcolor{teal}{The WR protocol} \textst{is an open hardware/software technology that improves the Gigabit Ethernet (GbE) and}  \textcolor{teal}{extends} Precise Time Protocol version 2 (PTPv2) with extra messages, proposed to be included in the new PTP release (PTPv3) as High Accuracy profile.\textst{for optical fiber links. Its main goal is}Its main goal is to provide a \textst{timing} synchronization \textst{with an} accuracy better than one nanosecond and \textst{the} precision in the scale of picoseconds. \textcolor{teal}{The major improvements introduced in the WR protocol address weak aspects of the PTPv2 protocol: the limitation in the phase difference measurements to one period of the system clock (8 ns); and the assumption of symmetry between the transmission and reception paths.}

\textst{WR is based on other technologies such as the L1 synthonization of Synchronous Ethernet (SyncE), an extension of PTPv2 (WR-PTP) and additional phase alignment techniques. The L1 synthonization is responsible for transmitting the master clock inside the data stream in order to slave devices can recover it from the network and adjust their oscillator frequencies to follow the reference clock.} 

WR also uses additional phase alignment techniques as the Digital Dual Mixed Time Difference (DDMTD) that is a module responsible for measuring the phase between two clocks. This information is used to change the frequency of the local oscillator for the synchronization process.

%%Internal oscillators of the WR devices run locked to the master's reference clock of the network, allowing precise phase difference 
%%measurements between the master clock and the slave clock. 

\ftgnote{Topología de una red WR}WR implements mechanisms to ensure \textst{the} deterministic and reliable data transfer between a thousand of nodes connected with \ftglnote{A mi me dijeron que eran largas para unos y muy cortas para otros...}\textst{large} \textcolor{teal}{optical fiber} links up to 10 km. A typical WR network is arranged in tree topology. This is very common in timing \textcolor{teal}{networks} where time reference is propagated downlink from \textcolor{teal}{a} root device, known as Grand Master, that can be connected to a very stable clock, such as an atomic clock or a GPS receiver. Currently, new network topologies are under-study in WR to add some mechanisms for improving the redundancy and security. \ftglnote{TODO: adornar esto un poco... Decir de que va almenos en 3-4 palabras}An example of this research is \cite{jlgutierrez-paper-redundancy}.

The intermediate levels of the network \textcolor{teal}{spread the timing packets to the final nodes}. Those levels are composed of other devices such as WR Switches, WR-ZEN or WR-LEN that have several ports and can behave as PTPv2 Boundary Clocks (BC). Their different ports are divided on two types: one of them is the slave (uplink) and connects to the upper layer and the other ports are masters (downlinks) and they are charged to propagate the synchronization to the next level of the hierarchy. The nodes of the last level of the network are known as Slave devices that recover the clock signal of the link and synchronizes their local oscillators to provide synchronization for a specific application or facility.

\missingfigure{Añadir figura para mostrar la topología WR}

WR is designed to be use \ftglnote{to be fitted, lo veo mejor} in a Field Programmable Gate Array (FPGA) device \textcolor{teal}{due to its flexibility to use in designs in a continuous development state.}  \textst{and }The source code is mainly written in Hardware Description Languages (HDLs) such as VHDL or Verilog. \ftglnote{Esto quizás lo diría de otra forma, o añadiría referencias para dar peso a la afirmación, por ejemplo la discover de GSI para altera y otro para xilinx}Moreover, there are several platforms that can implement WR that ensures the vendor-independent feature of WR. The main Intellectual Property (IP) block is the WR PTP core (WRPC) for WR nodes and the Real Time Subsystem (RTS) for WR switches\ftglnote{Algo más de chicha quizás interese aquí}. The most common WR nodes are based on carrier boards such as SPEC or SVEC and can be plugged in a PCIe/VME socket of a conventional PC. Some WR nodes have a standalone mode but in this case they can not benefit of high level features usually provided by a PC CPU.

\klyonenote{Here, I am not sure if we have to describe a little the WR-ZEN because in next section will be discussed in detail. However, we have to present it and I think that maybe some information can be repeated.}
\ftgnote{Yo no lo haría, queda raro de hecho en esta sección.}

The WR Zynq Embedded Node (WR-ZEN) is a new generation board that includes a Xilinx Zynq System-on-Chip (SoC). The Zynq SoC is composed of a FPGA and a hard ARM dual core microprocessor that can run an application that controls the hardware directly or a standard Operating System such as Linux that can host many different kind of processes at the same time. The WR-ZEN has been proposed to be used in the SKA project to implement the PPS distribution system that will be discussed in detail in the next section.
